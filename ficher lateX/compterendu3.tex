\documentclass{report}

\usepackage[latin1]{inputenc}
\usepackage[T1]{fontenc}
\usepackage[francais]{babel}
\usepackage{setspace}
\usepackage{listings} 
\usepackage{hyperref}

\begin{document}
\lstset{language=C}

\title{Compte-rendu N 3 de projet 2048}
\author{Guillaume ALMYRE 
\and
Ga\"{e}tan CHAMBRES
\and
Allan MAHAZOASY
\and 
Chrystelle PETUREAU}
\date{09/03/2015}

\maketitle

\section*{Informations sur le travail en groupe:}
L'adresse du d\'{e}p\^{o}t : \url{https://github.com/projetL2/2048.git} \\Les identifiants sont :
\begin{description}
\item [username : ] projetL2
\item [password : ] IN4001ggc  (pour Guillaume, Ga\"{e}tan, Chrystelle)
\end{description}

\section*{Etat des fonctions:}
\begin{description}
Le probl\`{e}me de get-tile \'{e}tait en fait un probl\`{e}me avec "afficher" qui avait une double boucle dans l'autre sens de "get-tile".\\
Toutes les m\'{e}thodes fonctionnent parfaitement \`{a} ce jour. Nous avons une version jouable du 2048. Il reste \`{a} optimiser le code en question.
\end{description}

\section*{Tests effectu\'{e}s:}
\begin{description}
\item [afficher, set-tile et get-tile:] Premi\`{e}res vraies fonctions de cr\'{e}\'{e}es, elles sont vitales pour tester les autres. Depuis le d\'{e}but du projet, elles ont servit un nombre incalculable de fois sans jamais r\'{e}v\'{e}ler aucun probl\`{e}me. Donc, elles fonctionnent parfaitement.\\
\item [new-grid :] On l'a test\'{e} dans le ficher test principal. Elle fait partie aussi de la base des tests des autres fonctions et n'a r\'{e}v\'{e}l\'{e} aucun probl\`{e}me \`{a} ce jour.\\
\item [add-tile :] Son test a consist\'{e} principalement \`{a} associer new-grid, afficher et add-tile dans une boucle for pour v\'{e}rifier son aspect al\'{e}atoire.\\
\item [copy-grid :]Associ\'{e}e \`{a} afficher, elle nous semble tout \`{a} fait identique.\\
\item [delete-grid :] On fait un fichier test avec new-grid, add-tile, copy-grid puis delete-grid. Ce fichier a \'{e}t\'{e} lancer avec valgrind pour v\'{e}rifier les \'{e}ventuelles fuites de m\'{e}moire.\\
\item [do-move :] Elle a \'{e}t\'{e} test\'{e}e dans une boucle for avec l'utilisation de la fonction random.\\
\item [grid-score et can-move :] Sont test\'{e}es \`{a} chaque boucle.\\

\item [play et game-over:] Un fichier particulier a \'{e}t\'{e} cr\'{e}er pour tester les possibilit\'{e}s de jeux. On a utilis\'{e} la biblioth\`{e}que NCURSES pour cela.  
\end{description}

\section*{Objectifs pour la s\'{e}ance du 17/03 :}
\begin{itemize}

\item Allan a cr\'{e}\'{e} un algorithme de gestion des cases vides, on doit l'ins\'{e}rer au code principal.\\

\item mettre dans les cases les puissances de 2 et non le nombre directement.

	
\end{itemize}


\end{document}