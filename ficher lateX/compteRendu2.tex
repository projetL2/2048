\documentclass{report}

\usepackage[latin1]{inputenc}
\usepackage[T1]{fontenc}
\usepackage[francais]{babel}
\usepackage{setspace}
\usepackage{listings} 
\usepackage{hyperref}

\begin{document}
\lstset{language=C}

\title{Compte-rendu N�1 de projet 2048}
\author{Guillaume ALMYRE 
\and
Ga�tan CHAMBRES
\and
Chrystelle PETUREAU}
\date{09/03/2015}

\maketitle

\section*{Information sur le travail en groupe:}
L'adresse du d�p�t : \url{https://github.com/projetL2/2048.git} \\Les identifiants sont :
\begin{description}
\item [username : ] projetL2
\item [password : ] IN4001ggc  (pout Guillaume, Ga�tan, Chrystelle)
\end{description}

\section*{Etat des fonctions sur les quelles on travail}
\begin{description}
\item [new-grid :] fonctionne parfaitement
\item [delete-grid :] faire un test pour les fuites de m�moire?
\item [copy-grid :] � tester
\item [grid-score :] fonctionne parfaitement
\item [get-tile :] � mettre � jour
\item [set-tile :] fonctionne parfaitement
\item [can-move :] � optimiser
\item [game-over :] � tester
\item [do-move :] tester le mouvement up, � optimiser
\item [add-tile :] fonctionne
\item [play :]
\item [afficher:] fonction parfaitement

\end{description}


\section*{Objectif pour la s�ance du 10/03 :}
\begin{itemize}

\item trouver un moyen de r�cup�rer les commandes de l'utilisateur pour d�placer la grille en fonction pour mettre � jour la fonction play.
\item oppitmiser can-move et do-move
\item mettre dans les cases les puissance de 2 et non le nombre directement
\item mettre � jour le fichier test. 
	
\end{itemize}


\end{document}