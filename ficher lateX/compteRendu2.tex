\documentclass{report}

\usepackage[latin1]{inputenc}
\usepackage[T1]{fontenc}
\usepackage[francais]{babel}
\usepackage{setspace}
\usepackage{listings} 
\usepackage{hyperref}

\begin{document}
\lstset{language=C}

\title{Compte-rendu N�1 de projet 2048}
\author{Guillaume ALMYRE 
\and
Ga�tan CHAMBRES
\and
Chrystelle PETUREAU}
\date{09/03/2015}

\maketitle

\section*{Informations sur le travail en groupe:}
L'adresse du d�p�t : \url{https://github.com/projetL2/2048.git} \\Les identifiants sont :
\begin{description}
\item [username : ] projetL2
\item [password : ] IN4001ggc  (pour Guillaume, Ga�tan, Chrystelle)
\end{description}

\section*{Etat des fonctions:}
 \\ Nous n'avons pas fait un r�sum� �tapes par �tapes car la phase de tests est finalement beaucoup plus longue que nous l'aurions imagin�e. Au fur et � mesure des diff�rents tests, on corrige le code pour qu'il soit op�rationnel, clair et optimis�. Voici donc l'�tat des fonctions � ce jour.\\

\begin{description}
\item [new-grid :] fonctionne parfaitement
\item [afficher:] fonctionne parfaitement
\item [grid-score :] fonctionne parfaitement
\item [set-tile :] fonctionne parfaitement
\item [delete-grid :] faire un test pour  v�rifier les fuites de m�moire
\item [copy-grid :] � tester (pour �tre sur d'avoir test� l'int�gralit� des fonctions).
\item [game-over :] � tester (pour �tre sur d'avoir test� l'int�gralit� des fonctions)
\item [get-tile :] il faut qu'on trouve un moyen de changer le return -1. De plus, on pense que peut �tre elle "lit" les donn�es � l'envers X pour les ligne et Y pour les colonnes.
\item [can-move :] on a test� les d�placements qui sont tous correct maintenant. On utilise les accesseurs pour un code plus propre.
\item [do-move :] on doit tester le mouvement UP. Il semble que can-move lui transmettait une erreur. on va v�rifier tous �a mardi 10 mars.
\item [add-tile :] on doit l'optimiser avec par exemple une liste de case vide.
\item [play :] on doit d'abords r�cup�rer les commandes de l'utilisateur pour pouvoir la tester.

\end{description}


\section*{Objectifs pour la s�ance du 10/03 :}
\begin{itemize}

\item trouver un moyen de r�cup�rer les commandes de l'utilisateur. Puis les utiliser pour d�placer les tiles de la grille. Enfin, mettre � jour la fonction play avec ces commandes l�.\\
\item optimiser le code avec une liste de cases vides.\\
\item mettre dans les cases les puissance de 2 et non le nombre directement.\\
\item mettre � jour le fichier test.\\
\item essayer de comprendre pourquoi get-tile a besoin des cordonn�es "� l'envers" pour fonctionner correctement.
	
\end{itemize}


\end{document}